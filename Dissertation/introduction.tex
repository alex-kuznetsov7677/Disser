\chapter*{Введение}                         % Заголовок
\addcontentsline{toc}{chapter}{Введение}    % Добавляем его в оглавление
Слабо столкновительная плазма с анизотропным распределением частиц по скоростям является неравновесной~\cite{Mikhailovsky1971,Krall1975}, и развивающиеся кинетические неустойчивости формируют в ней хаотические электромагнитные поля и согласованную с ними плазменную турбулентность. Ее динамика и свойства определяются нелинейными эффектами, которые во многих случаях хотя и являются не сильно выраженными, но для полноценного описания требуют трудоемких расчетов кинетики огромного числа частиц (при этом моделирование наиболее эффективным методом частиц в ячейках~\cite{Kato2005,Borodachev2010,Ruyer2015,Lazar2022,Borodachev2016_Radiofiz,Romanov2004} вносит численный шум, неизбежно искажающий результаты). Такая ситуация характерна для широкого круга задач физики разреженной космической, лазерной и газоразрядной плазмы, в которой время свободного пробега частиц много больше времени развития подобной слабо нелинейной турбулентности~\cite{Baumjohann2012,Treumann2009,Marcowith2016,Gary1993}. 

Среди неустойчивостей анизотропной плазмы апериодическая неустойчивость вейбелевского типа~\cite{Weibel1959,Zhou2022,Fried1959,Kalman1968,Morse1971,Kocharovsky2016,Lazar2006,Stockem2009,SchaeferRolffs2006} обладает одним из наибольших инкрементов и вместе с тем не сопровождается сильными резонансными нелинейными эффектами, поскольку ограничивается формированием квазимагнитостатических филаментов тока и не приводит к непосредственному возбуждению каких-либо волн. Настоящее исследование посвящено нелинейной стадии ее развития и описанию эволюции спектра возникающей турбулентности в простейших постановках 1- и 2-мерных задач на основе разработанного авторами численного кода, реализующего квазилинейный подход к расчету динамики вейбелевских мод~\cite{Kuznetsov2022}. Он учитывает основные нелинейные явления в указанной задаче и позволяет сразу находить представляющий физический интерес спектр полей и токов, избегая моделирования кинетики частиц. Анализ этого спектра актуален для физики звездного ветра, ударных волн в космической плазме, токовых структур, возникающих при лазерной абляции, и др.; см., например,~\cite{Lazar2022,Romanov2004,Medvedev2005,Chatterjee2017}.

В линейном приближении вейбелевская неустойчивость изучена достаточно подробно, особенно для бимаксвелловского распределения частиц~\cite{Weibel1959,Fried1959,Vagin2014}. Существующая полностью аналитическая квазилинейная теория эволюции вейбелевской турбулентности разработана лишь в одномерной (1D2V) геометрии, причем для весьма ограниченной области параметров и без должного описания временной эволюции~\cite{Pokhotelov2011}. Полуаналитическое решение системы квазилинейных уравнений в 1D3V геометрии~\cite{Ruyer2015} с опорой на эмпирические данные численного моделирования также применимо лишь для небольшой области параметров плазмы. 

В развиваемом численном квазилинейном подходе функция распределения частиц и электрическое и магнитное поля представлены в виде сумм пространственных мод (гармоник), удовлетворяющих самосогласованным квазилинейным уравнениям, в которых все нелинейные явления обусловлены совместным действием мод на форму средней по пространству функции распределения частиц по скоростям. Последняя определяет текущие значения инкрементов (декрементов) и, возможно, действительных частот всех рассматриваемых мод, в остальном эволюционирующих независимо. В результате, в отличие от метода частиц в ячейках, кардинально снижается уровень шумов и удается получать спектры вейбелевской турбулентности в гораздо более высоком качестве и в недоступных ранее областях параметров, правда, ценой потери некоторых слабых нелинейных эффектов при использовании сравнимых или даже б\'{о}льших вычислительных ресурсов.

Для определенности в конкретных расчетах ниже будем выбирать начальную функцию распределения частиц по скоростям бимаксвелловской, считая температуру частиц наибольшей вдоль оси $y$, называемой осью анизотропии. Для простоты будем предполагать плазму и все поля в ней однородными вдоль этой оси, т.е. решать систему уравнений Максвелла~-- Власова~\cite{Baumjohann2012} в одном (по координате $x$) или двух (по координатам $x$ и $z$) измерениях, а следовательно, полагать нулевой проекцию волновых векторов мод $\vec{k}$ на ось анизотропии: $k_y=0$. При этом в каждой моде электрическое поле $\vec{E}$ направлено вдоль оси анизотропии, а магнитное $\vec{B}$ ортогонально ей и волновому вектору $\vec{k}$ (ТЕМ-моды). 

Главная цель представленной работы состоит в изучении нелинейных явлений квазилинейного типа, доминирующих в процессе развития вейбелевской турбулентности. Насколько нам известно, последовательный квазилинейный анализ ее эволюции до сих пор никем не проводился ни для какой анизотропии начальной функции распределения частиц по скоростям (ср., например,~\cite{Ruyer2015,Pokhotelov2011,Davidson1972}). Более того, другими авторами не проводилось даже достаточно длительное моделирование динамики спектра вейбелевских мод в простейшей постановке задачи об одномерной (1D2V) и аксиально симметричной двумерной (2D3V) турбулентности, рассматриваемых в настоящей работе. Вместе с тем, некоторые выявленные нами особенности эволюции спектра и динамики отдельных мод аналогичны численно найденным ранее в других постановках задачи о вейбелевской турбулентности.

Следует отметить, что полноценное (3D3V) долговременное компьютерное моделирование изучаемых турбулентных явлений пока невозможно из-за недостаточной мощности вычислительных ресурсов. В ограниченных расчетах методом частиц в ячейках, проведенных в данной работе и ранее, далеко не всегда удается выделить слабые нелинейные эффекты, например, четырехволновое взаимодействие мод, на фоне более сильных квазилинейных эффектов и трудно отделимых неизбежных компьютерных шумов. Подобное выделение стало возможным только недавно и осуществлено в единичных случаях при специальных условиях для нестандартных задач (см.~\cite{Garasev2017_Radiophys,Garasev2021}), так что ниже оно затрагивается лишь вскользь.
