\chapter*{Введение}                         % Заголовок
\addcontentsline{toc}{chapter}{Введение}    % Добавляем его в оглавление
\newcommand{\me}{m_\mathrm{e}}
\newcommand{\wpl}{\omega_\mathrm{p}}

Слабо столкновительная плазма с анизотропным распределением частиц по скоростям является неравновесной~\cite{Mikhailovsky1971,Krall1973}, и развивающиеся кинетические неустойчивости формируют в ней хаотические электромагнитные поля и согласованную с ними плазменную турбулентность.
Ее динамика и свойства определяются нелинейными эффектами, которые во многих случаях хотя и являются не сильно выраженными, но для полноценного описания требуют трудоемких расчетов кинетики огромного числа частиц (при этом моделирование наиболее эффективным методом частиц в ячейках~\cite{Kato2005,Borodachev2010,Ruyer2015,Lazar2022,Borodachev2016_Radiofiz,Romanov2004} вносит численный шум, неизбежно искажающий результаты).
Такая ситуация характерна для широкого круга задач физики разреженной космической, лазерной и газоразрядной плазмы, в которой время свободного пробега частиц много больше времени развития подобной слабо нелинейной турбулентности~\cite{Baumjohann2012,Treumann2009,Marcowith2016,Gary1993}. 

Среди неустойчивостей анизотропной плазмы апериодическая неустойчивость вейбелевского типа~\cite{Weibel1959,Zhou2022,Fried1959,Kalman1968,Morse1971,Kocharovsky2016,Lazar2006,Stockem2009,SchaeferRolffs2006} обладает одним из наибольших инкрементов и вместе с тем не сопровождается сильными резонансными нелинейными эффектами, поскольку ограничивается формированием квазимагнитостатических филаментов тока и не приводит к непосредственному возбуждению каких-либо волн.
Настоящая работа посвящена анализу квазимагнитостатической турбулентности, обусловленной апериодической неустойчивостью вейбелевского типа~\cite{Weibel1959,Fried1959,Kocharovsky2016}, которая обладает одним из наибольших инкрементов среди различных неустойчивостей неравновесной анизотропной плазмы. 

Неустойчивость, впервые предсказанная Вейбелем в 1959 году~\cite{Weibel1959}, имеет простую физическую интерпретацию, предложенную Фридом~\cite{Fried1959}. В своей работе он рассмотрел суперпозицию двух противоположно направленных потоков холодной плазмы в присутствии слабого, периодического в пространстве магнитного поля (одной из гармоник шумов, присущих реальной плазме): электроны из встречных потоков смещались под действием силы Лоренца в разные стороны, что приводило к образованию токовых филаментов, которые в свою очередь способствовали экспоненциальному росту магнитного поля. 

Дисперсионный анализ апериодической неустойчивости вейбелевского типа в анизотропной плазме без внешнего магнитного поля проведён весьма подробно~\cite{Weibel1959,Zhou2022,Fried1959,Kalman1968,Morse1971,Kocharovsky2016,Lazar2006,Stockem2009,SchaeferRolffs2006}. Этой неустойчивости подвержены исключительно TM-моды, т.е. моды, магнитное поле которых ортогонально плоскости, определяемой их волновым вектором и осью анизотропии бимаксвелловской плазмы~\cite{Vagin2014}. Наибольший линейный инкремент наблюдается у мод с волновым вектором, перпендикулярным к оси анизотропии, т.е. лежащим в плоскости $xy$, и поэтому называемых поперечными. По мере уменьшения угла $\theta$ между волновым вектором и осью анизотропии, инкремент ТМ-мод, теперь уже наклонных, довольно быстро спадает до нуля.% \cite{???}. 

В присутствии внешнего однородного магнитного поля выделяют две ветви шланговых мод, отвечающие электронным так называемым "периодической" и "апериодической" шланговым неустойчивостям. 
Первая ветвь неустойчивости имеет место для волновых векторов, направленных преимущественно вдоль оси анизотропии, и инкремент таких мод много меньше их частоты, а область параметров, в которой неустойчивость развивается, является сравнительно узкой.%~\cite{??}. 
Другая ветвь шланговой неустойчивости, развивающейся значительно быстрее, реализуется в более обширной области параметров и под б\'{о}льшими углами к оси анизотропии ~\cite{Li2000,Camporeale2008,Lazar2013,Moya2022}. 
Анализ дисперсионного соотношения показывает, что в общем случае моды этой ветви являются смешанными (не ТМ и не ТЕ), хотя при стремлении величины внешнего поля  к нулю, $B_{ext}\rightarrow0$, эта ветвь является ничем иным как дисперсионной ветвью апериодически неустойчивых вейбелевских ТМ-мод. 
С увеличением внешнего магнитного поля угол между волновым вектором моды с наибольшим инкрементом, т.е. оптимальным волновым вектором $\overrightarrow{K}_{opt}$ , и осью анизотропии уменьшается. 
Ниже в настоящей работе, во избежание недоразумений, при наличии внешнего магнитного поля как вейбелевские, так и "апериодические" шланговые моды называются шланговыми модами.

Развитие магнитной турбулентности в помещенной во внешнее однородное магнитное поле $\vec{B}_{ext}=(0, 0, B_{ext})$ бесстолкновительной плазме на временах, допускающих пренебрежение движением тяжелых ионов, описывается известными уравнениями  Максвелла--Власова. %\cite{???}. 
Они имеют следующий вид  для электрического $\vec{E}(\vec{r}, t)$ и магнитного $\vec{B}(\vec{r}, t)$ полей и функции распределения электронов $f(\vec{v},\vec{r}, t)$, зависящих от времени $t$, радиуса-вектора $\vec{r}=\left(x,y,z\right)$ и скорости $\vec{v}=\left(v_x,v_y,v_z\right)$
\begin{align}
     \label{eq:maxw1} 
    \nabla \times \vec{B}=\dfrac{1}{c}\dfrac{\partial \vec{E}}{\partial t}+\dfrac{4\pi}{c}\vec{j}, \\
    %
    \label{eq:maxw2}
    \nabla \times \vec{E}=-\dfrac{1}{c}\dfrac{\partial \vec{B}}{\partial t}, \\
    %
    \dfrac{\partial f}{\partial t}+\vec{v}\dfrac{\partial f}{\partial \vec{r}}-\dfrac{e}{\me} \left(\vec{E}+\dfrac{1}{c}\left[\vec{v},\vec{B}+\vec{B}_{ext}\right]\right) \dfrac{\partial f}{\partial \vec{v}}=0,
    \label{eq:Vlasov}
\end{align}
где $c$~-- скорость света в вакууме, $e$ и $\me$~-- величина заряда и масса электрона, $\vec{j}=-e\iiint^{+\infty}_{-\infty}\vec{v}f(\vec{v},\vec{r}, t) d^3\vec{v}$~-- плотность тока, $n=\iiint^{+\infty}_{-\infty}f(\vec{v},\vec{r}, t) d^3\vec{v}$~-- концентрация электронов, $n_0$~-- ее начальное однородное значение.
На протяжении большей части данной работы, а именно в главах 1,2 и 4 начальное распределение частиц для выбрано бимаксвелловским~(\ref{eq:bimax}). Для определенности в конкретных расчетах ниже будем выбирать считая энергию частиц наибольшей вдоль оси $y$, называемой осью анизотропии. 
\begin{equation}
\label{eq:bimax}
\Psi(\vec{\beta})=\dfrac{1}{\pi^{3/2}\beta_{\perp0}^2 \beta_{\|0} } \exp\left(-\dfrac{\beta_x^2+\beta_z^2}{\beta_{\perp0}^2}-\dfrac{\beta_y^2}{\beta_{\|0}^2}\right),
\end{equation}
где $\beta_{x,y,z}={v_{x,y,z}}/{c}$, т.\,е. $\vec{\beta}=\vec{v}/{c}$, $\beta_{\|0}=\sqrt{2nk_bT_\|/m_e}/c$, $\beta_{\perp0}=\sqrt{2nk_bT_\perp/m_e}/c$, где включена постоянная Больцмана $k_b$.

Неравновесной плазме, частицы которой испытали стохастическое ускорение под действием того или иного широкополосного электромагнитного излучения, например в звездном ветре или различных ударных волнах, свойственны бикаппа-распределения с $\kappa >  1$, двумерный срез которых вдоль оси анизотропии:  
\begin{equation}
\label{eq:kappa}
    \Psi(\vec{\beta})=\dfrac{n}{\pi\theta_{\perp}\theta_{\|} } \left(1+\dfrac{\beta_x^2}{\kappa\theta_{\perp}^2}+\dfrac{\beta_y^2}{\kappa\theta_{\|}^2}\right)^{-\kappa-1} ,
\end{equation}
либо продакт-бикаппа-распределения с $\kappa > 1/2$, 
\begin{equation}
\label{eq:productkappa}
\Psi(\vec{\beta}) = \dfrac{\Gamma^2(\kappa+1)}{\Gamma^2(\kappa+0.5)} \: \dfrac{n}{\pi\kappa\theta_{\perp}\theta_{\|}}\left(1+\dfrac{\beta_x^2}{\kappa\theta_{\perp}^2}\right)^{-\kappa-1} \!\left(1+\dfrac{\beta_y^2}{\kappa\theta_{\|}^2}\right)^{-\kappa-1};
\end{equation}
см.~\cite{Lazar2010, Livadiotis2017, Livadiotis2021,Pierrard2010}. Выше введены характерные скорости $\theta_{\perp,\|}=\beta_{\perp,\|}\left(1-1/\kappa\right)^{1/2}$ для бикаппа-распределения и $\theta_{\perp,\|}=\beta_{\perp,\|}\left(1-1/(2\kappa)\right)^{1/2}$ для продакт-бикаппа-распределения.

Для различных задач физики лабораторной и космической бесстолкновительной плазмы, включая солнечный ветер и магнитосферы звезд и планет, характерно наличие пучка энергичных заряженных частиц (электронов и/или ионов) в теплой плазме~\cite{Gary1993,Treumann1997,Marsch2006}. 
Даже если распределение частиц по скоростям в плазме и в пучке является изотропным максвелловским (в соответствующей системе отсчета), для совместной системы "плазма+пучок" распределение будет анизотропным~(\ref{eq:bp}).  
В результате, согласно дисперсионному анализу~\cite{Mikhailovsky1971,Fried1959,Krall1973,Tzoufras2006,Bret2010}, могут одновременно развиваться различные кинетические неустойчивости, прежде всего резонансная квазиэлектростатическая двухпотоковая неустойчивость и апериодическая квазимагнитостатическая неустойчивость вейбелевского типа, известные также как пучковая и филаментационная соответственно. 
При прохождении через фоновую плазму достаточно широкого пучка частиц, поперечный размер которого много больше плазменного скин-слоя, возникает обратный ток, так что $n_b\beta_{b}=n_s\beta_{s}$~\cite{Shukla2018,Jia2013,Karlick2008}, где $n_i$ и $\beta_i=v_i/c$~--- начальная концентрация и безразмерная направленная скорость фракций частиц: индексы $b$ и $s$ соответствуют фону и пучку. 
Поэтому используемое в работе обезразмеренное начальное распределение электронов по скоростям имеет вид
\begin{equation}
\label{eq:bp}
    \Psi_0(\vec{\beta})=\dfrac{n_b}{\pi\beta_T^2} \exp\left(-\dfrac{\beta_x^2}{\beta_T^2}-\dfrac{\left(\beta_y+\beta_{b}\right)^2}{\beta_T^2}\right)+\dfrac{n_s}{\pi\beta_T^2 } \exp\left(-\dfrac{\beta_x^2}{\beta_T^2}-\dfrac{\left(\beta_y-\beta_{s}\right)^2}{\beta_T^2}\right).  
\end{equation}

Хотя в линейном приближении вейбелевская неустойчивость изучена достаточно подробно, особенно для бимаксвелловского распределения частиц~\cite{Weibel1959,Fried1959,Vagin2014}, существующая полностью аналитическая квазилинейная теория эволюции вейбелевской турбулентности разработана лишь в одномерной (1D2V) геометрии, причем для весьма ограниченной области параметров и без должного описания временной эволюции~\cite{Pokhotelov2011}. 
Полуаналитическое решение системы квазилинейных уравнений в 1D3V геометрии~\cite{Ruyer2015} с опорой на эмпирические данные численного моделирования также применимо лишь для небольшой области параметров плазмы. 
Поэтому нелинейная эволюция вейбелевской турбулентности исследуется на практике преимущественно с использованием численного метода частиц в ячейках~\cite{Kato2005,Borodachev2010,Ruyer2015,Lazar2022,Borodachev2016_Radiofiz,Romanov2004,Garasev2017_Radiophys,Garasev2021}. 
Его преимуществом является полный учет нелинейных взаимодействий, а недостатками~--- высокий уровень шумов и необходимость использования больших вычислительных ресурсов. 

В настоящем исследовании предложен оригинальный квазилинейный подход для описания нелинейной стадии развития вейбелевской турбулентности. Он учитывает основные нелинейные явления в указанной задаче и позволяет сразу находить представляющий физический интерес спектр полей и токов, избегая моделирования кинетики частиц.
В развиваемом численном квазилинейном подходе функция распределения частиц и электрическое и магнитное поля представлены в виде сумм пространственных мод (гармоник), удовлетворяющих самосогласованным квазилинейным уравнениям, в которых все нелинейные явления обусловлены совместным действием мод на форму средней по пространству функции распределения частиц по скоростям.
Последняя определяет текущие значения инкрементов (декрементов) и, возможно, действительных частот всех рассматриваемых мод, в остальном эволюционирующих независимо.
В результате, в отличие от метода частиц в ячейках, кардинально снижается уровень шумов и удается получать спектры вейбелевской турбулентности в гораздо более высоком качестве и в недоступных ранее областях параметров, правда, ценой потери некоторых слабых нелинейных эффектов при использовании сравнимых или даже б\'{о}льших вычислительных ресурсов.

Главная цель представленной работы состоит в изучении нелинейных явлений квазилинейного типа в той области параметров турбулентности, в которой они доминируют, и определение границ этой области. Насколько нам известно, последовательный квазилинейный анализ ее эволюции до сих пор никем не проводился ни для какой анизотропии начальной функции распределения частиц по скоростям (ср., например,~\cite{Ruyer2015,Pokhotelov2011,Davidson1972}).
Более того, другими авторами не проводилось даже достаточно длительное моделирование динамики спектра вейбелевских мод в простейшей постановке задачи об одномерной (1D2V) и аксиально симметричной двумерной (2D3V) турбулентности, рассматриваемых в настоящей работе. 
Вместе с тем, некоторые выявленные нами особенности эволюции спектра и динамики отдельных мод аналогичны численно найденным ранее в других постановках задачи о вейбелевской турбулентности.

В первой главе развит спектральный квазилинейный подход к задаче о ТЕМ-вейбелевской неустойчивости в анизотропной бесстолкновительной плазме, учитывающий только интегральное нелинейное взаимодействие мод (пространственных гармоник) посредством их совместного изменения средней по пространству функции распределения частиц по скоростям. 
В рамках данного приближения получена замкнутая система уравнений для одномерной эволюции функции распределения частиц и мод электромагнитного поля в условиях, когда ось анизотропии этой функции, волновой вектор мод и их магнитное поле взаимно ортогональны друг к другу. 
Проведено сравнение численного решения этой системы уравнений с имеющимися результатами одномерной аналитической квазилинейной теории в области ее применимости. Установлено, что в простейшей начальной постановке одномерной и аксиально симметричной двумерной задач для бимаксвелловской плазмы квазилинейные явления оказываются определяющими на весьма длительной стадии нелинейного развития турбулентности. 
На основе проведенного анализа выявлен вклад основных нелинейных эффекты в эволюцию пространственного спектра вейбелевской турбулентности и изучены её свойства, включая автомодельный характер и четыре качественно различные стадии динамики неустойчивых мод с разными волновыми числами.

Во второй главе спектральный квазилинейный подход обобщен на случай двумерной аксиальной симметричной ТЕМ-вейбелевской турбулентности. 
Проведено сравнение численного решения этой системы уравнений с результатами двумерного моделирования методом частиц в ячейках, учитывающим и прямое четырехволновое взаимодействие мод. 
Отмечено, что на более позднем этапе её затухания и в более общей постановке задачи, в частности, при наличии достаточно сильного внешнего магнитного поля, наряду с квазилинейными явлениями может проявляться и непосредственное нелинейное взаимодействие мод. 

В третье главе при помощи того же квазилинейного приближения описано совместное развитие двухпотоковой (ленгмюровской) и вейбелевской кинетических неустойчивостей в плазме с пучком частиц. 
Показано, что развивающаяся вейбелевская турбулентность магнитного поля может значительно деформировать резонансную с ленгмюровскими волнами область распределения частиц по скоростям, существенно влияя тем самым на формирование и особенно затухание ленгмюровской турбулентности. Ленгмюровская турбулентность электрического поля, в свою очередь, способна существенно изотропизовать распределение частиц по скоростям, а следовательно, изменить инкременты, характер эволюции и уровень насыщения гармоник вейбелевской турбулентности.

В четвертой главе моделированием методом частиц в ячейках исследовано развитие турбулентности, обусловленной апериодической электронной неустойчивостью вейбелевского и/или шлангового типа в анизотропной магнитоактивной плазме с бимаксвелловским распределением частиц по скоростям с осью наибольшей температуры, направленной вдоль заданного однородного магнитного поля. 
Установлены интегральные, спектральные и динамические свойства формирующегося турбулентного магнитного поля, включая законы нарастания и затухания отдельных пространственных гармоник (мод), зависящие от величины внешнего магнитного поля. 
Прослежена нелинейная  эволюция спектра вейбелевской/шланговой турбулентности в условиях эффективного трехволнового взаимодействия мод, конкурирующего с их квазилинейным взаимодействием. 
Охарактеризованы сходства и отличия полученных численных результатов для модельной двумерной и реальной трехмерной турбулентности. 
Показано, что квазилинейное описание спектральной динамики подобной турбулентности является корректным только при достаточно слабом внешнем магнитном поле. 

В пятой главе исследуется найденная зависимость насыщающего вейбелевскую неустойчивость среднеквадратичного магнитного поля от начальной анизотропии бимаксвелловского и различных бикаппа-распределений (\ref{eq:kappa}-\ref{eq:productkappa}) частиц по скоростям.
Показано, что для небольших по сравнению с единицей параметров анизотропии величина насыщающего поля существенно зависит от параметра каппа продакт-бикаппа-распределения частиц, т.\,е. от его энергетического профиля.