\chapter*{Введение}                         % Заголовок
\addcontentsline{toc}{chapter}{Введение}    % Добавляем его в оглавление
\newcommand{\me}{m_\mathrm{e}}
\newcommand{\wpl}{\omega_\mathrm{p}}

В линейном приближении вейбелевская неустойчивость как в кинетическом подходе к описанию бесстолкновительной плазмы, так и в гидродинамическом приближении весьма подробно изучена; см., например,~\cite{Kocharovsky2016}. Изучение нелинейной стадии проводилось в ряде работ, в том числе в гидродинамическом приближении, прежде всего, в случае двухпучкового распределения частиц (например,~\cite{Romanov2004,Bychenkov2003}). Однако в более адекватном кинетическом подходе данная задача остается мало изученной. В настоящей работе с целью исследования процессов, происходящих на нелинейной стадии неустойчивости, проведено численное исследование эволюции связанных пространственных гармоник возмущений функции распределения (ФР) частиц и магнитного поля в анизотропной бесстолкновительной плазме в рамках двумерной постановки задачи с различными начальными анизотропными распределениями частиц по скоростям (рис. \ref{fig:GeomIsolines}). ТМ-вейбелевская неустойчивость насыщается, т.\,е. прекращается рост среднеквадратичного магнитного поля, тогда, когда оно в достаточной мере выравнивает средние значения продольной ($T_{\|}$, вдоль оси $y$) и поперечной ($T_\perp $, вдоль оси $x$ или в плоскости $x,z$ в зависимости от постановки задачи) эффективных температур, так что его присутствие, пространственная неоднородность и особенно понизившийся и тоже неоднородный уровень анизотропии $A={T_{\|}}/{T_{\perp}}-1$ исключают экспоненциальное нарастание каких-либо возмущений, в том числе крупномасштабных (см.,~например,~\cite{Borodachev2016_Radiofiz}). В дальнейшем уровень турбулентности медленно понижается и её пространственный спектр постепенно эволюционирует в длинноволновую область спектра.

Слабо столкновительная плазма с анизотропным распределением частиц по скоростям является неравновесной~\cite{Mikhailovsky1971,Krall1973}, и развивающиеся кинетические неустойчивости формируют в ней хаотические электромагнитные поля и согласованную с ними плазменную турбулентность. Ее динамика и свойства определяются нелинейными эффектами, которые во многих случаях хотя и являются не сильно выраженными, но для полноценного описания требуют трудоемких расчетов кинетики огромного числа частиц (при этом моделирование наиболее эффективным методом частиц в ячейках~\cite{Kato2005,Borodachev2010,Ruyer2015,Lazar2022,Borodachev2016_Radiofiz,Romanov2004} вносит численный шум, неизбежно искажающий результаты). Такая ситуация характерна для широкого круга задач физики разреженной космической, лазерной и газоразрядной плазмы, в которой время свободного пробега частиц много больше времени развития подобной слабо нелинейной турбулентности~\cite{Baumjohann2012,Treumann2009,Marcowith2016,Gary1993}. 

Среди неустойчивостей анизотропной плазмы апериодическая неустойчивость вейбелевского типа~\cite{Weibel1959,Zhou2022,Fried1959,Kalman1968,Morse1971,Kocharovsky2016,Lazar2006,Stockem2009,SchaeferRolffs2006} обладает одним из наибольших инкрементов и вместе с тем не сопровождается сильными резонансными нелинейными эффектами, поскольку ограничивается формированием квазимагнитостатических филаментов тока и не приводит к непосредственному возбуждению каких-либо волн.
Настоящая работа посвящена анализу квазимагнитостатической турбулентности, обусловленной апериодической неустойчивостью вейбелевского типа~\cite{Weibel1959, Fried1959, Kocharovsky2016}, которая обладает одним из наибольших инкрементов среди различных неустойчивостей неравновесной анизотропной плазмы. 

Неустойчивость, впервые предсказанная Вейбелем в 1959 году~\cite{Weibel1959}, имеет простую физическую интерпретацию, предложенную Фридом~\cite{Fried1959}. В своей работе он рассмотрел суперпозицию двух противоположно направленных потоков холодной плазмы в присутствии слабого, периодического в пространстве магнитного поля (одной из гармоник шумов, присущих реальной плазме): электроны из встречных потоков смещались под действием силы Лоренца в разные стороны, что приводило к образованию токовых филаментов, которые в свою очередь способствовали экспоненциальному росту магнитного поля. 

На протяжении большей части данной работы, а именно в главах 1,2 и 4 начальное распределение частиц для определенности выбрано бимаксвелловским~\ref{eq:bp}. 
\begin{equation}
\label{eq:bp}
    \Psi_0(\vec{\beta})=\dfrac{n_b}{\pi\beta_T^2} \exp\left(-\dfrac{\beta_x^2}{\beta_T^2}-\dfrac{\left(\beta_y+\beta_{b}\right)^2}{\beta_T^2}\right)+\dfrac{n_s}{\pi\beta_T^2 } \exp\left(-\dfrac{\beta_x^2}{\beta_T^2}-\dfrac{\left(\beta_y-\beta_{s}\right)^2}{\beta_T^2}\right).  
\end{equation}

Неравновесной плазме, частицы которой испытали стохастическое ускорение под действием того или иного широкополосного электромагнитного излучения, например в звездном ветре или различных ударных волнах, свойственно бикаппа-распределение или продакт-бикаппа-распределение, которые используются в 5 главе.  

Для различных задач физики лабораторной и космической бесстолкновительной плазмы, включая солнечный ветер и магнитосферы звезд и планет, характерно наличие пучка энергичных заряженных частиц (электронов и/или ионов) в теплой плазме~\cite{Gary1993,Treumann1997,Marsch2006}. 
Даже если распределение частиц по скоростям в плазме и в пучке является изотропным максвелловским (в соответствующей системе отсчета), для совместной системы "плазма+пучок" распределение будет анизотропным.  
В результате, согласно дисперсионному анализу~\cite{Mikhailovsky1971,Fried1959,Krall1973,Tzoufras2006,Bret2010}, могут одновременно развиваться различные кинетические неустойчивости, прежде всего резонансная квазиэлектростатическая двухпотоковая неустойчивость и апериодическая квазимагнитостатическая неустойчивость вейбелевского типа, известные также как пучковая и филаментационная соответственно. 


Настоящее исследование посвящено нелинейной стадии ее развития и описанию эволюции спектра возникающей турбулентности в простейших постановках 1- и 2-мерных задач на основе разработанного авторами численного кода, реализующего квазилинейный подход к расчету динамики вейбелевских мод~\cite{Kuznetsov2022}.
Он учитывает основные нелинейные явления в указанной задаче и позволяет сразу находить представляющий физический интерес спектр полей и токов, избегая моделирования кинетики частиц. 
Анализ этого спектра актуален для физики звездного ветра, ударных волн в космической плазме, токовых структур, возникающих при лазерной абляции, и др.; см., например,~\cite{Lazar2022,Romanov2004,Medvedev2005,Chatterjee2017}.

В линейном приближении вейбелевская неустойчивость изучена достаточно подробно, особенно для бимаксвелловского распределения частиц~\cite{Weibel1959,Fried1959,Vagin2014}. Существующая полностью аналитическая квазилинейная теория эволюции вейбелевской турбулентности разработана лишь в одномерной (1D2V) геометрии, причем для весьма ограниченной области параметров и без должного описания временной эволюции~\cite{Pokhotelov2011}. Полуаналитическое решение системы квазилинейных уравнений в 1D3V геометрии~\cite{Ruyer2015} с опорой на эмпирические данные численного моделирования также применимо лишь для небольшой области параметров плазмы. 

В развиваемом численном квазилинейном подходе функция распределения частиц и электрическое и магнитное поля представлены в виде сумм пространственных мод (гармоник), удовлетворяющих самосогласованным квазилинейным уравнениям, в которых все нелинейные явления обусловлены совместным действием мод на форму средней по пространству функции распределения частиц по скоростям. Последняя определяет текущие значения инкрементов (декрементов) и, возможно, действительных частот всех рассматриваемых мод, в остальном эволюционирующих независимо. В результате, в отличие от метода частиц в ячейках, кардинально снижается уровень шумов и удается получать спектры вейбелевской турбулентности в гораздо более высоком качестве и в недоступных ранее областях параметров, правда, ценой потери некоторых слабых нелинейных эффектов при использовании сравнимых или даже б\'{о}льших вычислительных ресурсов.

Для определенности в конкретных расчетах ниже будем выбирать начальную функцию распределения частиц по скоростям бимаксвелловской, считая температуру частиц наибольшей вдоль оси $y$, называемой осью анизотропии. Для простоты будем предполагать плазму и все поля в ней однородными вдоль этой оси, т.е. решать систему уравнений Максвелла~-- Власова~\cite{Baumjohann2012} в одном (по координате $x$) или двух (по координатам $x$ и $z$) измерениях, а следовательно, полагать нулевой проекцию волновых векторов мод $\vec{k}$ на ось анизотропии: $k_y=0$. При этом в каждой моде электрическое поле $\vec{E}$ направлено вдоль оси анизотропии, а магнитное $\vec{B}$ ортогонально ей и волновому вектору $\vec{k}$ (ТЕМ-моды). 

Главная цель представленной работы состоит в изучении нелинейных явлений квазилинейного типа, доминирующих в процессе развития вейбелевской турбулентности. Насколько нам известно, последовательный квазилинейный анализ ее эволюции до сих пор никем не проводился ни для какой анизотропии начальной функции распределения частиц по скоростям (ср., например,~\cite{Ruyer2015,Pokhotelov2011,Davidson1972}). Более того, другими авторами не проводилось даже достаточно длительное моделирование динамики спектра вейбелевских мод в простейшей постановке задачи об одномерной (1D2V) и аксиально симметричной двумерной (2D3V) турбулентности, рассматриваемых в настоящей работе. Вместе с тем, некоторые выявленные нами особенности эволюции спектра и динамики отдельных мод аналогичны численно найденным ранее в других постановках задачи о вейбелевской турбулентности.

Следует отметить, что полноценное (3D3V) долговременное компьютерное моделирование изучаемых турбулентных явлений пока невозможно из-за недостаточной мощности вычислительных ресурсов. В ограниченных расчетах методом частиц в ячейках, проведенных в данной работе и ранее, далеко не всегда удается выделить слабые нелинейные эффекты, например, четырехволновое взаимодействие мод, на фоне более сильных квазилинейных эффектов и трудно отделимых неизбежных компьютерных шумов. Подобное выделение стало возможным только недавно и осуществлено в единичных случаях при специальных условиях для нестандартных задач (см.~\cite{Garasev2018,Garasev2021}), так что ниже оно затрагивается лишь вскользь.

Развитие магнитной турбулентности в помещенной во внешнее однородное магнитное поле $\vec{B}_{ext}=(0, 0, B_{ext})$ бесстолкновительной плазме на временах, допускающих пренебрежение движением тяжелых ионов, описывается известными уравнениями  Максвелла--Власова~\cite{???}. Они имеют следующий вид  для электрического $\vec{E}(\vec{r}, t)$ и магнитного $\vec{B}(\vec{r}, t)$ полей и функции распределения электронов $f(\vec{v},\vec{r}, t)$, зависящих от времени $t$, радиуса-вектора $\vec{r}=\left(x,y,z\right)$ и скорости $\vec{v}=\left(v_x,v_y,v_z\right)$
\begin{align}
     \label{eq:maxw1} 
    \nabla \times \vec{B}=\dfrac{1}{c}\dfrac{\partial \vec{E}}{\partial t}+\dfrac{4\pi}{c}\vec{j}, \\
    %
    \label{eq:maxw2}
    \nabla \times \vec{E}=-\dfrac{1}{c}\dfrac{\partial \vec{B}}{\partial t}, \\
    %
    \dfrac{\partial f}{\partial t}+\vec{v}\dfrac{\partial f}{\partial \vec{r}}-\dfrac{e}{\me} \left(\vec{E}+\dfrac{1}{c}\left[\vec{v},\vec{B}+\vec{B}_{ext}\right]\right) \dfrac{\partial f}{\partial \vec{v}}=0,
    \label{eq:Vlasov}
\end{align}
где $c$~-- скорость света в вакууме, $e$ и $\me$~-- величина заряда и масса электрона, $\vec{j}=-e\iiint^{+\infty}_{-\infty}\vec{v}f(\vec{v},\vec{r}, t) d^3\vec{v}$~-- плотность тока, $N=\iiint^{+\infty}_{-\infty}f(\vec{v},\vec{r}, t) d^3\vec{v}$~-- концентрация электронов, $N_0$~-- ее начальное однородное значение. 

Первая глава работы содержит обсуждение уравнений и посвящена анализу вейбелевской турбулентности в одномерном случае, когда волновой вектор множества учитываемых коллинеарных неустойчивых мод направлен поперек оси анизотропии, т.\,е. вдоль оси $x$. Именно в этом направлении возбуждаются гармоники с наибольшими инкрементами. Для рассматриваемых ТМ-возмущений электрическое поле направлено вдоль оси анизотропии $y$, а магнитное -- вдоль оси $z$.  В данном разделе внимание сосредоточено на механизме возбуждения и насыщения гармоник ФР и магнитного поля, кратных исходной возбуждемой. Продемонстрирована квазилинейность задачи, обеспечивающая возможность пренебрежения высшими кратными гармониками магнитного поля при описании развития вейбелевской турбулентности. Проведено тщательное сравнение полученных численных результатов построенной квазилинейной одномерной модели вейбелевской турбулентности с более ранними аналитическими результатами работы~\cite{Pokhotelov2011}

Во второй главе используются аналогичные квазилинейные уравнения для двумерной задачи и приводятся результаты их решения в простейшем случае аксиальной симметрии, в котором температура частиц и все характеристики вейбелевской турбулентности изотропны в плоскости $xz$, т.е. спектр зависит только от радиальной компоненты $k$ волнового вектора. Эти результаты, полученные для случая двумерной аксиально симметричной турбулентности, сравниваются с полученными методом частиц в ячейках при помощи кода EPOCH~\cite{Arber2015}, учитывающего и более тонкие нелинейные эффекты, в том числе четырехволновое взаимодействие. 

В третье главе при помощи того же квазилинейного приближения описано совместное развитие двухпотоковой (ленгмюровской) и вейбелевской кинетических неустойчивостей в плазме с пучком частиц. Будет показано, что развивающаяся вейбелевская турбулентность магнитного поля может значительно деформировать резонансную с ленгмюровскими волнами область распределения частиц по скоростям, существенно влияя тем самым на формирование и особенно затухание ленгмюровской турбулентности. Ленгмюровская турбулентность электрического поля, в свою очередь, способна существенно изотропизовать распределение частиц по скоростям, а следовательно, изменить инкременты, характер эволюции и уровень насыщения гармоник вейбелевской турбулентности.

В четвертой главе проанализировано  изменение эволюции распределения частиц по скоростям, среднеквадратичного магнитного поля, его спектра в целом и отдельных гармоник в частности с увеличением величины внешнего магнитного поля от малого до почти подавляющего неустойчивость для представительного диапазона величин начальной анизотропии распределения частиц по скоростям. Основное внимание уделяется описанию  процессов нелинейного, в частности, трех- и четырехволнового взаимодействия между модами. В заключительном разделе суммируются общие свойства эволюции квазимагнитостатической турбулентности, генерируемой шланговой неустойчивостью во внешнем магнитном поле, и обсуждается возможная роль этой турбулентности в формировании корональных вспышек на Солнце и звездах поздних спектральных классов.

В пятой главе рассматривается аналогичная постановка задачи для\\ бикаппа-распределения~\cite{Livadiotis2017, Livadiotis2021}, выбранного в качестве начального. Оно свойственно неравновесной плазме, частицы которой испытали стохастическое ускорение под действием того или иного широкополосного электромагнитного излучения, например в звездном ветре или различных ударных волнах, и в качестве частного случая содержит бимаксвелловское распределение частиц по скоростям. Исследования нелинейной стадии вейбелевской неустойчивости для начального бикаппа-распределения ранее отсутствовали.



