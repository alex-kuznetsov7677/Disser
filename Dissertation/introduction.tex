\chapter*{Введение}                         % Заголовок
\addcontentsline{toc}{chapter}{Введение}    % Добавляем его в оглавление
\newcommand{\me}{m_\mathrm{e}}
\newcommand{\wpl}{\omega_\mathrm{p}}

{\bfseries Актуальность темы}

Работа посвящена фундаментальной проблеме филаментации электрических токов и развития квазимагнитостатической турбулентности в плазме с анизотропными распределениями частиц по скоростям. 
Неизбежная для такой плазмы апериодическая неустойчивость вейбелевского типа~\cite{Mikhailovsky1971,Krall1973} приводит к формированию случайного мелкомасштабного магнитного поля. 
Это поле существенно изменяет кинетические свойства и динамику плазмы, даже если турбулентность является слабой, т.е. не содержит сильно нелинейных локализованных структур~\cite{Baumjohann2012,Treumann2009,Marcowith2016,Gary1993}. 
Согласно имеющимся наблюдениям и экспериментам, подобные анизотропные распределения частиц и обусловленная ими магнитная турбулентность характерны для многих переходных процессов в космической и лабораторной плазме и реализуются, например, при расширении неравновесной плазмы в фоновую~\cite{Zhou2022,Kalman1968,Morse1971,Kocharovsky2016,Lazar2006,Stockem2009,SchaeferRolffs2006}. 
%Кинетическое моделирование и анализ свойств данной турбулентности актуален для разнообразных задач физики звездного ветра, магнитосфер звезд и планет, бесстолкновительных ударных волн, инерционного термоядерного синтеза, микроволнового газового разряда, абляции материалов лазерными импульсами, генерации мегагауссного магнитного поля при лазерно-плазменном взаимодействии и др.

В настоящее время доминирующим подходом в исследовании этой, так называемой вейбелевской, турбулентности является численное описание с помощью метода частиц в ячейках. 
Расчеты с его использованием требуют больших вычислительных затрат и обычно сопровождаются высоким уровнем шумов. 
В то же время, как правило, они подтверждают тот факт, что вейбелевская турбулентность является слабой, а значит, может быть рассмотрена в квазилинейном приближении. 
В таком приближении эволюция усредненной по пространству компоненты функции распределения частиц по скоростям определяется интегральным нелинейным действием всех мод (пространственных гармоник) вейбелевской турбулентности. 
В свою очередь, их комплексные частоты задаются текущей формой этого распределения, т.е. динамика мод формально является линейной. 
Подобный квазилинейный подход является общим в теории волновой турбулентности и хорошо известен для высокочастотных волн – электромагнитных, плазменных, магнитогидродинамических. 
В контексте апериодической (вейбелевской) турбулентности, где фактически волн нет, существующая квазилинейная теория эволюции вейбелевской турбулентности разработана лишь в одномерной (1D2V) геометрии, причем для весьма ограниченной области параметров и без должного описания временной эволюции~\cite{Pokhotelov2011}. 
Полуаналитическое решение системы квазилинейных уравнений в 1D3V геометрии~\cite{Ruyer2015} с опорой на эмпирические данные численного моделирования также применимо лишь для небольшой области параметров плазмы.
В то же время существует ряд задач, в каждой из которых анализ квазимагнитостатической турбулентности с использованием квазилинейного описания может существенно прояснить осоенности её эволюции.  

Одной из подобных фундаментальных задач является анализ формированиия квазистационарных магнитных полей в анизотропной бесстолкновительной или слабостолнкновительной плазме и согласованной с ними плазменной турбулентности.
В частности, открытыми остаются вопросы об условиях генерации и пространственной структуре быстро эволюционирующего сильного магнитного поля и его корреляции с локальной анизотропией распределения электронов по скоростям, механизме их взаимосвязи, а также зависимости между макроскопическими кинетическими параметрами плазмы и параметрами формирующихся квазимагнитостатических полей. 
Подобные задачи особенно важны для современной физики лазерной плазмы и связанной с ней лабораторной астрофизикой, где при абляции различных мишеней используются мощные фемтосекундные импульсы, позволяющие разогревать только электроны, но не ионы.

Вопросы формированиия и эволюции квазистационарных магнитных полей в анизотропной плазме в присутствие внешнего магнитного поля особенно малоизуены: существующие результаты основываются на физичеки недостаточных двумерных (2D3V) расчетах~\cite{Camporeale2008,Hellinger2014,Lazar2023,Lopez2020} методом частиц в ячейках.
В то же время задача их изучения крайне актуальна как для астрофизических проблем, например, касающихся физики областей формирования  солнечного (звездного) ветра и магнитосфер звезд и планет, так и для проблем физики лабораторной плазмы, например, лазерной, получаемой при абляции мишени или струи газа фемтосекундными импульсами.

До сих пор слабо освещена проблема нелинейного взаимодействия волновой ленгмюровской и квазимагнитостатической турбулентности в бесстолкновительной плазме с распределениями частиц «плазма - пучок», важная для анализа таких явлений, как формирование бесстолкновительных ударных волн и структур в аккреционных дисках и колонках, взаимопроникновения соседних облаков и потоков частиц звёздного ветра, развитие корональных выбросов массы и солнечных вспышек, нагрев плазмы и изменение её кинетических свойств при инжекции пучков высокоэнергичных частиц и др.~\cite{Marcowith2016,Aschwanden2005,Medvedev2006,Nishikawa2009,Kato2007}. 
Имеющиеся выборочные работы в этом направлении исследований основаны преимущественно на расчетах методом частиц в ячейках, зачастую ограничиваются гидродинамическим режимом той или иной неустойчивости и по существу не касаются, а тем более не детализируют взаимное влияние ленгмюровской и вейбелевской  турбулентности; ср., например,~\cite{Kong2009,Ruyer2015,Bret2010,Lazar2023} (при наличии внешнего магнитного поля см. также работы~\cite{Lazar2023,Lopez2020}). 
Так, в обзоре~\cite{Bret2010}, являющемся, по-видимому, наиболее полным для задачи о взаимосвязи ленгмюровской и квазимагнитостатической турбулентностей в отсутствие внешнего магнитного поля, обсуждаются условия насыщения и возможные нелинейные локализованные структуры в турбулентности каждого типа, но не изучается влияние изменения функции распределения частиц под действием турбулентности одного типа на развитие другой, хотя и упоминается возможность их последовательного развития.

Совершенно не изученными являются нелинейные свойства вейбелевской неустойчивости и формируемой ей квазимагнитостатической турбулентности, возникающих в плазме с немаксвелловскими распределениями частиц по скоростям.
Вместе с тем неравновесной плазме, частицы которой испытали стохастическое ускорение под действием того или иного широкополосного электромагнитного излучения, например в звездном ветре или различных ударных волнах свойственны распределения с надтепловым высокоэнергетическим «хвостом», который может быть удобно описан каппа-распределениями.

{\bfseries Цели и задачи работы}

Общей целью диссертационной работы является построение приближенной теории магнитной турбулентности на основе квазилинейного подхода, разработка оригинальной компьютерной программы для ее расчета и создание ясной физической картины эволюции пространственного спектра этой турбулентности в однородной плазме при различных начальных распределениях частиц, в том числе с учетом внешнего магнитного поля или электростатической турбулентности.
Задачами являются следующие
\begin{enumerate}
    \item Аналитическая разработка альтернативного описания магнитной турбулентности при помощи приближенных квазилинейных уравнений для широкого ансамбля мод слабостолкновительной магнитоактивной плазмы с произвольным начальным распределением частиц по скоростям и возможностью учета однородной в пространстве инжекции частиц с анизотропным распределением по скоростям.
    
    \item Построение оригинальной численной программы для эффективного решения квазилинейных уравнений в пространственно двумерных задачах при учете всех трех компонент скорости частиц, их столкновений, а также магнитного поля.
    
    \item Сопоставление полученных таким образом результатов с результатами проведенных в рамках проекта расчетов методом частиц в ячейках и анализ сходств и отличий в них, прежде всего, по отношению к динамике среднеквадратичного магнитного поля, характерного волнового числа турбулентности и параметра анизотропии плазмы, в том числе при наличии внешнего магнитного поля, инжекции частиц и их столкновений.
    
    \item Выяснение влияния внешнего магнитного поля, пучковой неустойчивости плазмы (прежде всего в отношении ленгмюровских волн) на  
    \begin{enumerate}
        \item величину среднего квадрата насыщающего неустойчивость магнитного поля,  
        \item долговременную нелинейную эволюцию среднеквадратичного магнитного поля,
        \item законы роста/спадания его гармоник и их нелинейного (резонансного) взаимодействия, 
        \item форму усредненной самосогласованной функции распределения частиц по скоростям, 
        \item параметр её анизотропии, в том числе как функцию начальной величины анизотропии.
    \end{enumerate}

    \item Демонстрация возможности согласованного квазилинейного описания апериодической вейбелевской и волновой ленгмюровской турбулентности и изучение их взаимного влияния в плазме с пучковыми распределениями частиц по скоростям. 

    \item Нахождение пределов применимости развитой квазилинейной теории магнитной турбулентности в бесстолкновительной анизотропной магнитоактивной плазме.

    \item Изучение зависимости среднего квадрата насыщающего магнитного поля и автомодельных эволюционных свойств вейбелевской турбулентности от вида функции распределения частиц и величины её начальной анизотропии, а также от внешнего магнитного поля.
\end{enumerate}

{\bfseries Методы исследования}

Основным методом исследования является численное решение самосогласованных уравнений Власова– Максвелла с граничными и начальными условиями, отвечающими поставленным физическим задачам.
Численное моделирование осуществлялось при помощи оригинального кода, разрабатанного на основе квазилинейного подхода к уравнениям Власова-Максвелла, а также методом частиц в ячейках при помощи кода EPOCH~\cite{Arber2015}. 
Сравнительный анализ полученных решений проводился с использованием метода дисперсионных уравнений, параметрического представления функций, усреднения по углу и других методов современной теоретической физики.

{\bfseries Научная новизна}

В диссертационной работе развит оригинальный квазилинейный численный подход к описанию эволюции квазимагнитостатической турбулентности и соответствующей деформации функции распределения частиц, использующий приближение слабо нелинейной турбулентности, т.е. слабого взаимодействия ее пространственных мод (гармоник). 
Этот подход и развитые с его помощью представления о квазимагнитостатической турбулентности являются универсальными, т.е. применимы к различным исходным функциям распределения частиц по скоростям, включая не только бимаксвелловские, но и бикаппа-, комбинированные пучковые, не осесимметричные и другие распределения, представляющие интерес в различных ситуациях в плазме, включая наличие внешнего магнитного поля.

Посредством совместного анализа результатов квазилинейного моделирования и стандандартного для подобных задач метода частиц в ячейках выявлена область применимости квазилинейного подхода, а значит, доминирования интегрального нелинейного взаимодействия между модами, а также получены многочисленные зависимости между характеристиками квазимагнитостатической турбулентности в различных задачах. 

{\bfseries Теоретическая и практическая значимость}

Научная значимость работы состоит, во-первых, в расширении квазилинейной теории с известного случая волновой (высокочастотной) неустойчивости на нетривиальный случай не волновой, а по существу апериодической неустойчивости и, во-вторых, в применении этой теории к задачам физики магнитной турбулентности в лазерной плазме, получаемой абляцией мишеней фемтосекундными оптическими импульсами, и в астрофизической плазме, формируемой во вспышках на звездах поздних спектральных классов и в идущем от них звездном ветре. 

{\bfseries Основные положения, выносимые на защиту}

\begin{enumerate}
    \item В начальной постановке одномерной и аксиально симметричной двумерной задач для бимаксвелловской плазмы при достаточно слабом внешнем магнитном поле квазилинейные явления являются определяющими на весьма длительной стадии нелинейного развития турбулентности. 
    Общий сценарий эволюции спектра вейбелевской турбулентности представляется следующим. 
    При небольшом уровне начальных шумов на линейной стадии неустойчивости анизотропной плазмы происходит экспоненциальный, апериодический рост мод в довольно широком интервале волновых чисел, для которых инкремент не сильно меньше максимального. 
    К моменту насыщения роста среднеквадратичного магнитного поля профиль спектра может значительно сузиться и центральная группа его мод, следуя квазилинейной динамике и за счет <<квадратичной>> нелинейности возбуждая вторые гармоники функции распределения частиц и деформируя среднюю по пространству форму этой функции, начинает существенно уменьшать средний параметр анизотропии и инкремент всех мод, причем для указанных центральных мод последний становится даже отрицательным. 
    В результате коротковолновое крыло спектра постепенно затухает, а длинноволновое продолжает расти, причем там темп роста амплитуд мод почти сразу после момента насыщения становится примерно степенным (вместо экспоненциального) с показателем, зависящим от волнового числа моды: более длинноволновые моды нарастают медленнее менее длинноволновых. 
    В ходе процесса максимальными амплитудами по очереди начинают обладать все более длинноволновые моды, и величина этого максимума медленно убывает.

    \item В начальной постановке трехмерной задачи для незамагниченной бимаксвелловской плазмы помимо слабых эффектов трехволновой генерации ТЕ-моды и четырехволновой генерации коротковолновых мод возможна генерация наклонных к оси анизотропии ТМ-мод, а значит, увеличивается разнообразие нелинейных взаимодействий. 
    Вследствие этого, значительно усилению затухания среднеквадратичного турбулентного магнитного поля в сравнении с аксиально симметричной двумерной задачей, несмотря на то что общий сценарий эволюции спектра вейбелевской турбулентности остается тем же, а значит, преимущественно управляемым квазилинейным взаимодействием.
    При достаточно сильном внешнем магнитном поле поперечные к нему и к оси анизотропии моды могут быть подавлены, а наклонные оставаться неустойчивыми. 
    В этом случае трехволновое взаимодействие проявляется в генерации поперечных к оси анизотропии линейно устойчивых мод, а также мод с продольными компонентами оптимального волнового числа примерно кратными продольной компоненте наиболее неустойчивой моды. 
    
    \item  При достаточно сильном внешнем магнитном поле, существенно деформирующем область линейной неустойчивости относительно случая незамагниченной плазмы, ключевую роль в спектральной динамике играют трехволновое и четырехволновое взаимодействие между модами.
    Управляемый этими и другими прямымыми нелинейными взаимодействиями между модами, диапазон наиболее энергонесущих мод по мере эволюции турбулентности смещается в область малых волновых чисел, устойчивую относительно линейных возмущений.
    В квазилинейных расчетах, из которых трехволновое и четырехволновое взаимодействия исключены, этот процесс невозможен, а значит, наблюдается качественно отличная картина временной эволюции спектра турбулентности. 
    Следовательно, квазилинейные расчеты дают корректное представление о динамике спектра магнитной турбулентности в присутствии внешнего магнитного поля лишь при настолько слабом его значении, что закон дисперсии не отличим от случая незамагниченной плазмы. 
    \textcolor{blue}{В противном случае существует продолжительный промежуток нелинейной эволюции вейбелевской турбулентности в течение которого суммарное влияние трехволновых и четырехволновых взаимодействий может быть эффективно учтено в квазилинейном подходе путем ввода столкновительного члена вида БГК, физически соответствующего рассеянию частиц турбулентным магнитным полем, известным также как "аномальное".} 

    \item  Вейбелевская турбулентность магнитного поля может значительно деформировать резонансную с ленгмюровскими волнами область распределения частиц по скоростям, существенно влияя тем самым на формирование и особенно затухание ленгмюровской турбулентности.
     Ленгмюровская турбулентность электрического поля, в свою очередь, способна существенно изотропизовать распределение частиц по скоростям, а следовательно, изменить инкременты, характер эволюции и уровень насыщения гармоник вейбелевской турбулентности.
    
    \item Для небольших по сравнению с единицей параметров анизотропии величина насыщающего поля существенно зависит от параметра каппа продакт-бикаппа распределения частиц, т.\,е. от его энергетического профиля.
\end{enumerate}

{\bfseries Достоверность результатов}

Проведенные исследования опираются на известные физические модели, широко используемые при изучении квазистационарных процессов в слабостолкновительной плазме и основанные на уравнениях Власова– Максвелла, а также на ряде общетеоретических методов, имеющих строгое математическое обоснование: теорию возмущений, спектральные разложения, метод усреднения по большим интервалам и другие. 
Полученные аналитические оценки и результаты численного моделирования согласованы с экспериментальными данными и теоретическими результатами других научных групп.


{\bfseries Публикации и апробация результатов}

Основные результаты работы докладывались на следующих конференциях и научных школах:

- International Conference «Frontiers of Nonlinear Physics» (Н. Новгород, 2024 г.)

- XVIII–XXII научные школы «Нелинейные волны» (Н. Новгород,2018–2024 г.)

- «Всероссийская астрономическая конференция- 2024» (Нижний Архыз, 2024 г.)

– International Conferences «Shilnikov Workshop» (Н. Новгород, 2023 г.),– международная конференция 

- «Лазерные, плазменные исследования и технологии ЛаПлаз-2023» (Москва, 2023 г.)

- конференции «Физика плазмы в Солнечной системе» (Москва, 2022–2026 г.),

- COSPAR 2022, 44th Scientific Assembly (Афины, Греция, 2022 г.),

– XXXI International Astronomical Union General Assembly (Пусан, Республика Корея, 2022 г.)

- 37th European Conference on Laser Interaction with Matter (Лиссабон, Португалия, 2024 г.)

{\bfseries Личный вклад автора}

Все основные теоретические результаты, изложенные в диссертации, получены лично автором либо при его непосредственном участии.
Численные програмы для эффективного решения квазилинейных уравнений были разработаны лично автором. 
Постановка начальных задач для численного моделирования осуществлялась автором при консультативной поддержке со стороны научного руководителя. 
Расчеты методом частиц в ячейках проводились совместно с Гарасёвым М.А.
Анализ магнитоактивной анизотропной бимаксвелловской плазмы в линейном приближении выполнен совместно с Емельяновым Н.А.
Количественная обработка и теоретический анализ результатов расчетов осуществлены автором при консультативной поддержке со стороны научного руководителя.

{\bfseries Структура и объем диссертации}
Диссертация состоит из введения, пяти глав, заключения и списка литературы. Полный объем диссертации составляет ??? страниц, включая ?? рисунка.
Список литературы содержит ??? наименований.
