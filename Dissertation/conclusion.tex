\chapter*{Заключение}                       % Заголовок
\addcontentsline{toc}{chapter}{Заключение}  % Добавляем его в оглавление

%% Согласно ГОСТ Р 7.0.11-2011:
%% 5.3.3 В заключении диссертации излагают итоги выполненного исследования, рекомендации, перспективы дальнейшей разработки темы.
%% 9.2.3 В заключении автореферата диссертации излагают итоги данного исследования, рекомендации и перспективы дальнейшей разработки темы.
%% Поэтому имеет смысл сделать эту часть общей и загрузить из одного файла в автореферат и в диссертацию:

 Основные результаты диссертационной работы заключаются в следующем.

\begin{enumerate}
    \item Развит спектральный квазилинейный подход к задачам об одномерной, аксиально симметриченой двумерной ТЕМ- и двумерной ТМ-вейбелевской неустойчивости в анизотропной бесстолкновительной плазме, учитывающий только интегральное нелинейное взаимодействие мод (пространственных гармоник) посредством их совместного изменения средней по пространству функции распределения частиц по скоростям.
    В вырожденных случаях одномодовой и двумодовой эволюции показана возможность пренебречь прямым нелинейным межмодовым взаимодействием в незамагниченной плазме.
    Получена замкнутая система уравнений для эволюции взаимодействующих пространственных гармоник возмущений функции распределения частиц и магнитного поля для одномерного случая, а также обоих двумерных случаев, включая аксиально симметричную двумерную ТЕМ-вейбелевскую турбулентность, когда ось анизотропии этой функции, волновой вектор мод и их магнитное поле взаимно ортогональны друг к другу, и двумерную ТМ-вейбелевскую турбулентность, когда ось анизотропии плазмы и волновые векторы гармоник, а также векторы электрического поля в них лежат в одной плоскости расчета
    
    \item С помощью стандартного метода Стёрмера~-- Верле (Leapfrog) в одномерном и обоих двумерных случаях разработана программа для получения численных решений соответствующей квазилинейной системы уравнений.
    Проведены многократные сравнения численных решений этих систем уравнений с имеющимися результатами одномерной аналитической квазилинейной теории в области ее применимости, а также с результатами моделирования методом частиц в ячейках, учитывающим и прямое четырехволновое взаимодействие мод. 

    \item Установлено, что в начальной постановке одномерной и аксиально симметричной двумерной задач для бимаксвелловской плазмы квазилинейные явления оказываются определяющими на весьма длительной стадии нелинейного развития турбулентности, а общий сценарий эволюции представляется следующим.
    При небольшом уровне начальных шумов на линейной стадии неустойчивости анизотропной плазмы происходит экспоненциальный, апериодический рост мод. 
    К моменту насыщения роста среднеквадратичного магнитного поля профиль спектра значительно сужается и центральная группа его мод, следуя квазилинейной динамике и за счет <<квадратичной>> нелинейности деформируя среднюю по пространству форму этой функции, начинает существенно уменьшать инкремент всех мод, причем для указанных центральных мод последний становится даже отрицательным. 
    В результате коротковолновое крыло спектра постепенно затухает, а длинноволновое продолжает расти, причем там закон роста амплитуд непосредственно послое насыщения становится приблизительно степенным (вместо экспоненциального или нелинейно индуцированного) с показателем, зависящим от волнового числа моды: более длинноволновые моды нарастают медленнее менее длинноволновых. 
    В ходе этого процесса максимальными амплитудами по очереди начинают обладать все более длинноволновые моды, причем волновое число, отвечающее данному максимуму, уменьшается обратно пропорционально времени в небольшой степени (для аксиально симметричной турбулентности в сильно анизотропной плазме --- приблизительно обратно пропорционально корню квадратному от времени $t^{-0.5}$). 
    Закон затухания амплитуд $b_{K_{n}}$ коротковолновых мод со временем приблизительно одинаков и (после усреднения по аксиальному углу) тоже оказывается примерно степенным, не сильно зависящим от первоначального параметра анизотропии, тогда как показатель степенного роста длинноволновых мод значительно сильнее зависит от этого параметра и от волнового числа моды. 
    В аксиально симметричной двумерной задаче с $\beta_{\perp0}=0.1$ при $A_{0}=10$ указанные показатели лежат в диапазонах от -1.3 до -1 и от 1 до 2 соответственно для коротковолновых и длинноволновых мод, а показатель медленного примерно степенного спадания среднеквадратичного магнитного поля после момента насыщения близок к $-0.3$.
    
    \item  Проведен анализ совместного развития двухпотоковой (ленгмюровской) и филаментационной (вейбелевской) кинетических неустойчивостей в плазме с пучком частиц на основе вышеописанного квазилинейного подхода, учитывающего интегральное нелинейное взаимодействие пространственных мод (гармоник), обусловленное изменением средней по пространству функции распределения частиц по скоростям. 
    А именно, в рамках начальной двумерной задачи для ряда параметров плазмы и пучка проведено численное исследование эволюции спектров ленгмюровской (плазменной) и вейбелевской (магнитной) турбулентности. 
    Установлено, что развивающаяся вейбелевская турбулентность магнитного поля может значительно деформировать резонансную с ленгмюровскими волнами область распределения частиц по скоростям, существенно влияя тем самым на формирование и особенно затухание ленгмюровской турбулентности. 
    Ленгмюровская турбулентность электрического поля, в свою очередь, способна существенно изотропизовать распределение частиц по скоростям, а следовательно, изменить инкременты, характер эволюции и уровень насыщения гармоник вейбелевской турбулентности. 
    Найдены характерные особенности совместной динамики обоих турбулентных спектров и указаны необходимые условия их взаимовлияния.

    \item Моделированием методом частиц в ячейках исследовано развитие турбулентности, обусловленной апериодической электронной неустойчивостью вейбелевского и/или шлангового типа в анизотропной магнитоактивной бесстолкновительной плазме с аксиально симметричным распределением частиц по скоростям. 
    Рассмотрена начальная задача для бимаксвелловского распределения с осью наибольшей температуры, направленной вдоль заданного однородного магнитного поля. 
    Установлены интегральные, спектральные и динамические свойства формирующегося турбулентного магнитного поля, включая законы нарастания и затухания отдельных пространственных гармоник (мод), зависящие от величины внешнего магнитного поля. 
    Прослежена нелинейная  эволюция спектра вейбелевской/шланговой турбулентности в условиях эффективного трехволнового взаимодействия мод, конкурирующего с их квазилинейным взаимодействием. 
    Охарактеризованы сходства и отличия полученных численных результатов для модельной двумерной и реальной трехмерной турбулентности. 
    Показано, что квазилинейное описание спектральной динамики подобной турбулентности является корректным только при достаточно слабом внешнем магнитном поле

    \item Посредством сравнения результатов квазилинейных расчетов и симуляций методом частиц в ячейках, учитывающих также прямое нелинейное взаимодействие между модами, в незамагниченной плазме выявлены такие проявления последнего как трехволновая генерация наклонных мод и ТЕ-мод.
    В магнитоактивной плазме выявлена нелинейная генерация поперечных мод, включающая случай сильных магнитных полей, при которых данные моды устойчивы согласно линейному дисперсионному уравнению.
    Кроме того, в двумерных симуляциях наблюдалась мод с волновыми векторами, удвоенными в сравнении с векторами наиболее неустойчивых мод, а в трехмерных симуляциях~--- генерация гармоник с проекциями волнового вектора на ось анизотропии, удвоенными в сравнении с соответствующей проекцией волнового вектора наиболее неустойчивой моды. 
    Выявленное наличие нелинейно индуцированных мод объяснено с привлечением трехволнового взаимодействия на основе анализа геометрии области неустойчивости, наблюдаемой величины инкрементов и поляризации мод.

    \item  С целью эффективного учета нелинейных взаимодействий в квазилинейных уравнениях в рамках БГК $\tau$-приближения введена частота аномальных столкновений. 
    Показано, что в слабо магнитоактивной плазме существует значительный промежуток эволюции квазимагнитостатической турбулентности, на протяжении которого спектр последней может быть корректно корректно получен из квазилинейного моделирования с аномальными столкновениями.

    \item  На основе численного решения квазлинейной системы уравнений, описывающих ТМ-вейбелевскую турбулентность незамагниченной плазмы найдена зависимость насыщающего среднеквадратичного магнитного поля от начальной анизотропии бимаксвелловского и различных бикаппа{\PunctumKappa}распределений частиц по скоростям. 
    Показано, что для небольших по сравнению с единицей параметров анизотропии величина насыщающего поля существенно зависит от параметра каппа продакт-бикаппа{\PunctumKappa}распределения частиц, т.\,е. от его энергетического профиля. 

\end{enumerate}

Автор выражает благодарность и большую признательность научному руководителю Кочаровскому Владимиру Владиленовичу за поддержку, помощь, обсуждение результатов и научное руководство.